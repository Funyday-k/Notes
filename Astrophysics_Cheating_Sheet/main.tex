\documentclass{sciposter}
\usepackage{lipsum}
\usepackage{epsfig}
\usepackage{amsmath}
\usepackage{amssymb}
\usepackage{multicol}
\usepackage{graphicx,url,physics}
\usepackage[portuges, brazil, english]{babel}   
\usepackage[utf8]{inputenc}
%\usepackage{fancybullets}
\newtheorem{Def}{Definition}


\title{Nome do Projeto}
%Título do projeto

\author{Nome do Aluno \& Nome do Orientador}
%nome dos autores

\institute 
{Bacharelado em Engenharia de Computação\\
CEFET/RJ - campus Petrópolis - Avenida do Imperador, 971}
%Nome e endereço da Instituição

\email{e-mail do aluno}
% Onde você coloca os emails dos integrantes


%\date is unused by the current \maketitle

\rightlogo[1]{logoEngComp}
\leftlogo[1]{logoCefetCampusPetropolis}
% Exibe os logos (direita e esquerda) 
% Procure usar arquivos png ou jpg, e de preferencia mantenha na mesma pasta do .tex
%%%%%%%%%%%%%%%%%%%%%%%%%%%%%%%%%%%%%%%%%%%%%%%%%%%%%%%%%%%%%%%%%%%%%%%%%%%%%%%%
%%% Begin of Document



\begin{document}
%define conference poster is presented at (appears as footer)

% \conference{{\bf SEPEX 2017}, Semana de Ensino, Pesquisa \& Extensão}

%\LEFTSIDEfootlogo  
% Uncomment to put footer logo on left side, and 
% conference name on right side of footer

% Some examples of caption control (remove % to check result)

%\renewcommand{\algorithmname}{Algoritme} % for Dutch

%\renewcommand{\mastercapstartstyle}[1]{\textit{\textbf{#1}}}
%\renewcommand{\algcapstartstyle}[1]{\textsc{\textbf{#1}}}
%\renewcommand{\algcapbodystyle}{\bfseries}
%\renewcommand{\thealgorithm}{\Roman{algorithm}}

% \maketitle

%%% Begin of Multicols-Enviroment
\begin{multicols}{3}

%%% Abstract

\section{Spectrum}

Y-axis$= E^{2}\frac{\dd{J}}{\dd{E}} $, means per \textbf{area} per \textbf{time} per \textbf{solid angle} per \textbf{energy}.
%%% Introduction
\section{Cross Section}
One paritcle interaction, 
\begin{itemize}
  \item $\sigma_{AB}n_{B}L \ll 1  $: very likely to pass through(\textbf{optically thin})
  \item $\sigma_{AB}n_{B}L \gg 1:  $ very likely to interact(\textbf{optically thick})
\end{itemize}

Probability:
\begin{align}
    P=1-e^{-n\sigma L} 
\end{align}
where $\tau=n\sigma L$ is called \textbf{Optical Depth}.

Unit: Barn $1 barn = 10^{-28}m^{2}  $.




\section{Diffusion Model}

Diffusion-loss equation,
\begin{align}
    \frac{\partial n}{\partial t}=\nabla \cdot \left(D \vec{\nabla}n\right)-\frac{\partial}{\partial E}(n\dot{E})+Q
\end{align}
Diffusion-Convection equation,
\begin{align}
    \frac{\partial}{\partial t}n=\nabla \cdot \left(D\vec{\nabla}n-\vec{V}n\right)-\frac{\partial }{\partial E}(n\dot{E})+Q
\end{align}
with momentum loss term $\dot{p}=-\frac{1}{3}(\nabla \cdot V)p$.

Rigity: $R=\frac{p}{q}$. Motivation: Lamor Radius is propotional to the rigity $r_{g}=\frac{p_{\bot } }{q} $.

Number of particles per phase space: $f=\frac{\dd{N}}{\dd[3]{p}\dd[3]{x}}$.

Differential number density of particles, $n=\frac{\dd{N}}{\dd{p}\dd[3]{x}}=4\pi p^{2}f $.

From diffusion-loss equations, we can imply $$D\frac{\partial f}{\partial r}+\frac{V_{p} }{3}\frac{\partial f}{\partial p}=0 \implies \dd{f}(r,p)=0 \implies f(r_{1},p_{1}  )=f(r_{1},p_{1}  )  $$ 
with definition of flux $I=v n/(4 \pi)=vp^{2}f  $, we have relation
\begin{align}
    \frac{I(p)}{v p^{2} }=\frac{I(p_{ILS} ) }{v_{LIS}p_{LIS}^{2}   }
\end{align}
combining with solar modulation potential $\phi$, we have 
\begin{align}
    \frac{I(p)}{v p^{2} }=\frac{I(p+\phi ) }{v_{lis}(p+\phi)^{2}    }
\end{align}

\section{CR Secondaries}
The full propagation euqation,

\begin{align}
    \begin{aligned}
      \frac{\partial \psi(r,p,t)}{\partial t}&=q(r,p,t)+\overset{\text{Diffusion Convection}}{\nabla\cdot (D_{xx}\nabla \psi-V\psi )}\\
      &+\underset{\text{Re-acceleration}}{\frac{\partial  }{\partial p}p^{2}D_{pp}\frac{\partial }{\partial p}\frac{1}{p^{2} }\psi  }-\underset{\text{Continuous Energy Loss}}{\frac{\partial }{\partial p}\left[\dot{p}\psi-\frac{p}{3}(\nabla\cdot V)\psi\right]}\\
      &-\underset{\text{Fragmentation\& Radi. decay Loss}}{\frac{1}{\tau_{f} }\psi-\frac{1}{\tau_{r} }\psi}
    \end{aligned}
\end{align}
For Fragmentation Loss, for the i-th species, it's loss by $\rightarrow$ j-th species 
\begin{align}
    \frac{\partial n_{i} }{\partial t}=-n_{i}(\frac{\rho}{m})_{ism}\sigma_{i\rightarrow j}v   
\end{align}
 For radioactive decay loss,
 \begin{align}
     \frac{\partial n_{i} }{\partial t}=-n_{i}\frac{1}{\tau_{i} }, \tau_{i} \text{\; is the lifetime}  
 \end{align}

For simple case: Leaky box approx, one species dominates the production, $Q=0$. We have relation,
\begin{align}
    \frac{n_{i} }{T_{e} }=-\frac{n_{i} }{T_{f} }-\frac{n_{i} }{T_{dec} }+C_{i} 
\end{align}
where $C_{i} $ is the production of "i" due to other species, then we can get expression of $n_{i} $,
\begin{align}
    n_{i}=\frac{C_{i} }{1/T_{e}+1/T_{f}+1/T_{dec}   } 
\end{align}




%%% References

%% Note: use of BibTeX als works!!


\end{multicols}

\end{document}