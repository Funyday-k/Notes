\newpage

\mainmatter
\pagenumbering{arabic}
\chapter{Introduction to Curved World}


\section{The principle of equivalence and its consequences}

Gravity = Geometry
\begin{itemize}
    \item How does the existence of matter / energy act to curve spacetime? - Einstein equation
    \item How do particles (including mass-less particles) travel in curved space-time, such that we call their trajectories being influenced by gravity?
\end{itemize}

Eötvös experiment :$m_{I}=m_{G}$, which implies \textbf{weak equivalence principle}.


\bbox{Strong equivalence principle}{
    At every point in arbitrary gravitational field, it is possible to choose a locally inertial coordinate system, such that (within a sufficiently small region) the laws of nature take the same form as in the unaccelerated Cartesian coordinate system
}

The action of gravity should be attributed to the curvature of space-time. No such thing as "global inertial frame". We focus on local inertial frames.

\section{Gravitational Redshift}

Because of the acceleration, Benny receives the signals when he is moving at a faster ray then when they were emitted.

For Benny position: $\begin{cases}z_{B}=\frac{1}{2}gt^{2}\\z_{A}=\frac{1}{2}gt^{2}+h\end{cases}$, and the time interval is $t_{1}=h-\frac{1}{2}gt_{1}^2$ 


A light wave at frequency $\nu$, $\nu \varpropto \Delta \tau^{-1}$. If light is emitted at freq. $v\nu_{*}$(forme the surface of star)

\begin{align}
    \nu_{*}=\nu(1-\frac{GM}{Rc^{2}})=\nu(1-\frac{\Delta \phi}{c^{2}})
\end{align}

\section{Mathematical description of curved space-time}

$\Delta \tau_{A} \neq \Delta \tau_{B} \implies $ space-time is curved in the presence of gravity. 

For equivalence principle: in a small enough region of space-time, the laws of physics are the same as in S.R.. Freely falling coordinate system :$\phi^{\alpha}$

The proper time: 
\begin{align}
    d\tau^{2}=-\frac{1}{c^{2}}\eta_{\alpha \beta}\dd{\xi^{\alpha}} \dd{\xi^{\beta}}
\end{align}

Freely falling particle:
\begin{align}
    \frac{\dd{U^{\alpha}}}{\dd{\tau}}=\frac{\dd[2]{\xi^{\alpha}}}{\dd{\tau}^{2}}=0
\end{align}

Express the proper time in arbitrary coordinate system $x^{\mu}$, for chain rule
\begin{align}
    \dd\tau^{2}&=-\frac{1}{c^{2}}\eta_{\alpha \beta}\left(\frac{\partial \xi^{\alpha}}{\partial x^{\mu}}\right)\dd{x^{\mu}} \left(\frac{\partial \xi^{\beta}}{\partial x^{\nu}}\right)\dd{x^{\nu}}\\
    &=-\frac{1}{c^{2}}\equiv \eta_{\alpha\beta}\left(\frac{\partial \xi^{\alpha}}{\partial x^{\mu}}\right)\left(\frac{\partial \xi^{\beta}}{\partial x^{\nu}}\right)
\end{align}

Then we can define the metric for arbitrary coordinate system:
\begin{align}
    g_{\mu\nu}\equiv \eta_{\mu\nu}\left(\frac{\partial \xi^{\alpha}}{\partial x^{\mu}}\right)\left(\frac{\partial \xi^{\beta}}{\partial x^{\nu}}\right)
\end{align}

In arbitrary coordinate system, a free-falling particle does same accelerate,
\begin{align}
    0&=\frac{\dd{U^{\alpha}}}{\dd{\tau}}=\frac{\dd}{\dd\tau}\left(\frac{\partial \xi^{\alpha}}{\partial x^{\mu}}\frac{\partial x^{\mu}}{\partial \tau}\right)\\
    0&=\left[\left(\frac{\partial ^{2}\xi^{\alpha}}{\partial x^{\mu}\partial x^{\nu}}\right)\left(\frac{\dd x^{\nu}}{\dd \tau}\right)\right]\frac{\dd x^{\mu}}{\dd\tau}+\left(\frac{\partial \xi^{\alpha}}{\partial x^{\mu}}\right)\left(\frac{\dd[2]{x^{\mu}}}{\dd\tau^{2}}\right)\\
    0&=\frac{\dd[2]{x^{\lambda}}}{\dd\tau}+\Gamma^{\lambda}_{\mu\nu}\frac{\dd{x^{\mu}}}{\dd\tau}\frac{x^{\nu}}{\dd\tau},
\end{align}
where $\Gamma^{\lambda}_{\mu\nu}$ is called \textbf{Affine connection}
\begin{align}
    \Gamma^{\lambda}_{\mu\nu}\equiv \frac{\partial x^{\lambda}}{\partial \xi^{\alpha}}\frac{\partial^{2}\xi^{\alpha}}{\partial x^{\mu}\partial ^{\nu}}
\end{align}

Then we can get \textbf{geodesic equation}

\begin{align}
    \frac{\dd U^{\lambda}}{\dd\tau}=-\Gamma^{\lambda}_{\mu\nu} U^{\mu}U^{\nu}
\end{align}

Then we want to find the connection between the Affine connection and the metric tensor

\begin{align}
    \frac{\partial g_{\mu\nu}}{\partial x^{\mu}}=\frac{\partial ^{2}\xi^{\alpha}}{\partial x^{\mu}\partial x^{\nu}}\eta_{\alpha\beta}+\frac{\partial \xi^{\alpha}}{\partial x^{\mu}}\frac{\partial^{2}\xi^{\beta}}{\partial x^{\nu}\partial x^{\lambda}}\eta_{\alpha \beta}
\end{align}
use $\frac{\partial^{2} \xi ^{\beta}}{\partial x^{\nu}\partial x^{\lambda}}=\Gamma^{\rho}_{\nu\lambda}\frac{\partial \xi^{\beta}}{\partial x^{\rho}}$
 
\begin{align}
    \frac{\partial g_{\mu\nu}}{\partial x^{\lambda}}=\Gamma^{\rho}_{\mu\lambda}\frac{\partial \xi^{\alpha}}{\partial x^{\rho}}\frac{\partial \xi^{\beta}}{\partial x^{\nu}}\eta_{\alpha\beta}+\Gamma^{\rho}_{\nu\lambda}\frac{\partial \xi^{\alpha}}{\partial x^{\mu}}\frac{\partial \xi^{\beta}}{\partial x^{\rho}}\eta_{\alpha \beta}=\Gamma^{\rho}_{\lambda\mu}g_{\rho\nu}+\Gamma^{\rho}_{\lambda\nu}g_{\rho\mu}
\end{align}

Integrate them, we can get the expression of Affine connection in terms of metric
\begin{align}
    \Gamma^{\sigma}_{\lambda\mu}=\frac{1}{2}g^{\nu \sigma}\left(\frac{\partial g_{\mu\nu}}{\partial x^{\lambda}}+\frac{\partial g_{\lambda \nu}}{\partial x^{\mu}}-\frac{\partial g_{\mu\lambda}}{\partial x^{\nu}}\right)
\end{align}

\bbox{Example in 2d flat space in $(r,\theta)$ coordinate}{
    \begin{align}
        &g_{\mu\nu}=\begin{pmatrix}
        1 & 0\\
        0 & r^{2}
        \end{pmatrix}\\
        &\Gamma^{r}_{\theta\theta}=-r\\
        &\Gamma^{\theta}_{r\theta}=\Gamma^{\theta}_{\theta r}=\frac{1}{r}
    \end{align}
}