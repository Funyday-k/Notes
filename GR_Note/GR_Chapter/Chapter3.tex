\newpage

\mainmatter
\pagenumbering{arabic}
\chapter{Introduction to Curved World}

\newpage

\section{The principle of equivalence and its consequences}

Gravity = Geometry
\begin{itemize}
    \item How does the existence of matter / energy act to curve spacetime? - Einstein equation
    \item How do particles (including mass-less particles) travel in curved space-time, such that we call their trajectories being influenced by gravity?
\end{itemize}

Eötvös experiment :$m_{I}=m_{G}$, which implies \textbf{weak equivalence principle}.


\bbox{Strong equivalence principle}{
    At every point in arbitrary gravitational field, it is possible to choose a locally inertial coordinate system, such that (within a sufficiently small region) the laws of nature take the same form as in the unaccelerated Cartesian coordinate system
}

The action of gravity should be attributed to the curvature of space-time. No such thing as "global inertial frame". We focus on local inertial frames.

\section{Gravitational Redshift}

Because of the acceleration, Benny receives the signals when he is moving at a faster ray then when they were emitted.

For Benny position: $\begin{cases}z_{B}=\frac{1}{2}gt^{2}\\z_{A}=\frac{1}{2}gt^{2}+h\end{cases}$, and the time interval is $t_{1}=h-\frac{1}{2}gt_{1}^2$ 




