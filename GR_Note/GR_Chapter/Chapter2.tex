\newpage

\mainmatter
\pagenumbering{arabic}
\chapter{Special Relativity and Classical Field Theory}

\section{Review of Special Relativity}

\subsection{Transformation}

If we have two frames of space $\mathcal{O} $ and $\mathcal{O} ^{'}$ with velocity $v$ and $u$ respectively, we want to change frame from $\mathcal{O}$ to $\mathcal{O}'$. How can we do it? For classical physics, we usually use Galilee Transformation:
\begin{align}
    \vec{u}'=\vec{u}+\vec{v}
\end{align}

But we also have to face some questions from new phenomena

\begin{itemize}
    \item What is an initial frame?
    \item It should be tested in high speed situation ($v\sim c$)
\end{itemize}

But for Lorentz transformation:

\begin{align}
    -(\delta \tau)^{2}=-(\delta t)^{2}+\frac{(\delta l)^{2}}{c^{2}}
\end{align}


\subsection{Physics in flat space: Special Relativity}

\begin{tcolorbox}[title=\textbf{Special Relativity},colback=SeaGreen!10!CornflowerBlue!10,colframe=RoyalPurple!55!Aquamarine!100!]
    The laws of nature are invariant in all inertial reference frames.

    Speed of light $c$ is constant.
\end{tcolorbox}


\bbox{Spacetime interval}{
    \begin{align}
        (\text{Interval})^{2}=(\delta s)^{2}=x^{2}+y^{2}+z^{2}-c^{2}t^{2}
    \end{align}
}

We introduce the four dimensional coordinate $x^{\mu}$

\begin{align}
    x^{\mu}=\begin{pmatrix}
    x^{0}\\
    x^{1}\\
    x^{2}\\
    x^{3}
    \end{pmatrix}=\begin{pmatrix}
    ct\\
    x\\
    y\\
    z
    \end{pmatrix}=\begin{pmatrix}
    ct\\
    x^{i}
    \end{pmatrix}
\end{align}

And metric is defined by

\begin{align}
    \eta_{\mu \nu}=\begin{pmatrix}
    -1 &  &  & \\
     & 1 &  & \\
     &  & 1 & \\
     &  &  & 1
    \end{pmatrix}_{GR}=-(\eta_{\mu\nu})_{QFT}
\end{align}

So the interval can be calculated by a metric and two four dimensional vectors

\begin{align}
    \Delta s^{2}=\begin{pmatrix}
    x^{0} &x^{1}  &x^{2}  &x^{3} 
    \end{pmatrix} \begin{pmatrix}
    -1 &  &  & \\
     & 1 &  & \\
     &  & 1 & \\
     &  &  & 1
    \end{pmatrix} \begin{pmatrix}
    x^{0}\\
    x^{1}\\
    x^{2}\\
    x^{3}
    \end{pmatrix}=\sum_{i=0}^{3}\eta_{\mu\nu}x^{\mu}x^{\nu}=x^{2}+y^{2}+z^{2}-c^{2}t^{2}
\end{align}

What are all the transformation which conserve the interval. For spatial rotations

\begin{align}
    R_{x}=\begin{pmatrix}
    1 &  &  & \\
     &  1&  & \\
     &  & \cos\theta &\sin \theta \\
     &  & \sin\theta &\cos\theta 
    \end{pmatrix}
\end{align}

But for Lorentz transformation, which is called "boost"

\begin{align}
    &x^{\mu'}=\Lambda^{\mu'}_{\nu}x^{\nu}\\
    &\Lambda^{\mu'}_\nu=\begin{pmatrix}
    \cosh \omega & -\sinh \omega &  & \\
    -\sinh \omega &\sinh \omega  &  & \\
     &  & 1 & \\
     &  &  & 1
    \end{pmatrix}
\end{align}

So we can prove that interval is Lorentz invariant:
\begin{align}
    \begin{aligned}
        \Delta s^{2}&=\eta_{\mu\nu}'\Delta x^{\mu'}x^{\nu'}\\
        &=\eta_{\mu\nu}'\Lambda^{\mu'}_\sigma \Delta x^{\sigma}\Lambda^{\nu'}_{\kappa} \Delta x^{\kappa}\\
        &=\eta_{\sigma \kappa}\Delta x^{\sigma}\Delta x^{\kappa}
    \end{aligned}
\end{align}

For two definitions which are covariant vectors and contravariant vectors are
\begin{itemize}
    \item Covariant vector: transformed as basis.
    \item Contravariant vector: transformed opposite to the basis like: $x^{\mu}, p^{\mu}$.
\end{itemize} 

\bbox{Proper time $\tau$}{The time as measured by an observer comoving with the clock. That doesn't see the clock moving with respect to it.}


\begin{align}
    \eta_{\mu\nu} V^{\mu}V^{\nu}=\begin{cases}
        <0,\; \text{timelike}\\
        =0,\; \text{null/lightlike}\\
        >0,\; \text{spcaelike}
    \end{cases}
\end{align}


\subsection{Special Relativity Kinematics}

\subsection{Maxwell Equation}

We have learned that Maxwell equations before

\begin{align}
    \begin{aligned}
        \nabla \cdot E &=\frac{\rho}{\epsilon_{0}}\\
        \nabla\cdot B&=0\\
        \nabla \times E&=-\frac{\partial B}{\partial t}\\
        \nabla \times B&=\mu_{0}j+\mu_{0}\epsilon_{0}\frac{\partial E}{\partial t}
    \end{aligned}
\end{align}

Use Einstein convention, we can express

\begin{align}
    \begin{aligned}
        \epsilon^{ijk}\partial_{i}B_{k}-\partial_{0}E^{i}&=4\pi J^{i}\\
        \partial_{i}E^{i}&=4\pi J^{0}\\
        \epsilon^{ijk}\partial_{j}E_{k}+\partial_{0}B^{i}&=0\\
        \partial_{i}B^{i}&=0
    \end{aligned}
\end{align}

Then we try to construct electromagnetic field tensor

\begin{align}
    F^{\mu\nu}=\begin{pmatrix}
    0 & -E_{1} &-E_{2}  &-E_{3} \\
    E_{1} & 0 & B_{3} & -B_{2}\\
    E_{2} & -B_{3} &  0& B_{1}\\
    E_{3} & B_{2} & -B_{1} &0 
    \end{pmatrix}=-F_{\mu\nu}
\end{align}

From components, we have 

\begin{align}
    \partial_{\nu}F^{\mu\nu}=4\pi J^{\mu}
\end{align}

\section{Classical Field Theory}




